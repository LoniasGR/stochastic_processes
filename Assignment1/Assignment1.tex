\documentclass{article}

    % Input language encoding
    \usepackage[utf8]{inputenc}
   
    % Output languages
    \usepackage[english, greek]{babel}
    \usepackage{alphabeta}
    \usepackage{csquotes}
    
    % Fonts
    \usepackage[T1,LGR]{fontenc}
    \usepackage{lmodern}

    % Images
    \usepackage{graphicx}
    \usepackage{float}
    \usepackage{caption}
    \usepackage{subcaption}

    % Bibliography 
    \usepackage{biblatex}
    
    % Add animated images
    \usepackage{animate}

    % Frames
    \usepackage{framed}

    % Math
    \usepackage{amsmath}
    \usepackage{amssymb}

    % Paragraph Formatting
    \usepackage{parskip}

    % Code
    \usepackage{listings}
    \usepackage{fancyvrb}

    % Different Enumerations
    \usepackage{enumitem}

    % Trees
    \usepackage{qtree}

    % Other Drawings
    \usepackage{tikz}
    \usetikzlibrary{shapes,backgrounds}

    % Links
    \usepackage{hyperref}

    % Color
    \usepackage{color}
   
    % Setup

    % For hyperlinks
    \hypersetup{
        colorlinks=true,
        linkcolor=blue,
        filecolor=magenta,      
        urlcolor=cyan,
    }

    \urlstyle{same}
    
    % For code
    \definecolor{codegreen}{rgb}{0,0.6,0}
    \definecolor{codegray}{rgb}{0.5,0.5,0.5}
    \definecolor{codepurple}{rgb}{0.58,0,0.82}
    \definecolor{backcolour}{rgb}{0.95,0.95,0.92}
     
    \lstdefinestyle{mystyle}{
        backgroundcolor=\color{backcolour},   
        commentstyle=\color{codegreen},
        keywordstyle=\color{magenta},
        numberstyle=\tiny\color{codegray},
        stringstyle=\color{codepurple},
        basicstyle=\fontsize{8}{11}\selectfont\ttfamily,
        breakatwhitespace=false,         
        breaklines=true,                 
        captionpos=b,                    
        keepspaces=true,                 
        numbers=left,                    
        numbersep=5pt,                  
        showspaces=false,                
        showstringspaces=false,
        showtabs=false,                  
        tabsize=4
    }

    \lstset{style=mystyle}


    % For math
    \DeclareMathSizes{10}{10}{10}{10}
    \setlength{\parindent}{0cm}

    % Foreign Language macro
    \newcommand{\english}[1]{\foreignlanguage{english}{{#1}}}

    % Images folder
    \graphicspath{ {./plots/} }

    % Bibliography source
    %\addbibresource{report.bib}



    \title{Στοχαστικές Διαδικασίες \\
    1ή Θεωρητική Άσκηση}

\begin{document}

\pagenumbering{gobble}
\date{}
\author{Λεωνίδας Αβδελάς $|$ ΑΜ: 03113182}

\maketitle

\section*{Άσκηση 4}

Έχουμε τον πίνακα:
\begin{equation*}
    P = 
    \begin{pmatrix}
    0 & 3/4 & * & 0 & 0\\
    3/4 & 0 & 0 & 1/8 & *\\
    1/2 & 1/4 & 1/4 & * & *\\
    * & 3/5 & 1/5 & 1/5 & *\\
    0 & 0 & 1/10 & 1/5 & *\\
    \end{pmatrix}
\end{equation*}

Για να είναι ο πίνακας πίνακας πιθανοτήτων μετάβασης, θα πρέπει για κάθε $p(i,j)$ να ισχύει $\sum_{j}{p(i,j)} = 1$.

Άρα για την πρώτη γραμμή: $0 + \frac{3}{4} + * + 0 + 0 = 1$, οπότε $* = \frac{1}{4}$.

Για την δεύτερη γραμμή: $\frac{3}{4} + 0 + 0 + \frac{1}{8} + * = 1$, οπότε $* = \frac{1}{8}$.

Για την τρίτη γραμμή: $\frac{1}{2} + \frac{1}{4} + \frac{1}{4} + * + * = 1$, οπότε και για τα δύο $* = 0$.

Για την τέταρτη γραμμή: $* + \frac{3}{5} + \frac{1}{5} + \frac{1}{5} + * = 1$, οπότε και για τα δύο $* = 0$.

Για την πέμπτη γραμμή: $0 + 0 + \frac{1}{10} + \frac{1}{5} + * = 1$, οπότε $* = 7/10$.

\section*{Άσκηση 5}

Δεδομένου ότι ο χώρος καταστάσεων είναι $\mathbb{X} = \{K, B, Σ, Y, M\}$, ο πίνακς πιθανοτήτων μετάβασης θα είναι:

\begin{equation*}
    P = 
    \begin{pmatrix}
        0 & 1/2 & 1/2 & 0 & 0\\
        1/2 & 0 & 1/2 & 0 & 0\\
        1/4 & 1/4 & 0 & 1/4 & 1/4\\
        0 & 0 & 1/2 & 0 & 1/2\\
        0 & 0 & 1/2 & 1/2 & 0\\
    \end{pmatrix}
\end{equation*}

\section*{Άσκηση 6}

Θέλουμε να μοντελοποιήσουμε την θέση του ρήγα, άρα ο χώρος καταστάσεων μας θα είναι $\mathbb{X} = \{Θ_1, Θ_2, Θ_3, Θ_4, Θ_5\}$.

Θα μελετήσουμε την θέση του ρήγα περιγραφικά πρώτα και μετά θα φτιάξουμε τον πίνακα πιθανοτήτων μετάβασης.

Αν είναι στην θέση $Θ_1$, έχει πιθανότητα να βρεθέι στην μέση $2/3$. Ακόμα,με πιθανοτήτα $1/3$ μπορεί να επιλεγεί το χαρτί στην $Θ_5$, οπότε σε αυτή την περίπτωση ο ρήγας μένει στην θέση του.

Αν είναι στην θέση $Θ_2$, το χαρτί στην $Θ_1$ έχει πιθανότητα $2/3$ να επιλεχθεί, οπότε σε αυτή την περίπτωση ο ρήγας θα μετακινηθεί στην $Θ_1$. Αν επιλεχθεί το χαρτί στην $Θ_5$, ο ρήγας μένει στην θέση του.

Αν είναι στην θέση $Θ_3$, το χαρτί στην $Θ_1$ έχει πιθανότητα $2/3$ να επιλεχθεί, οπότε ο ρήγας θα μετακινηθεί στην θέση $Θ_2$ και το χαρτί από την $Θ_1$ θα πάει στην $Θ_3$.  Αν επιλεχθεί το χαρτί στην $Θ_5$, ο ρήγας θα μετακινηθεί στην $Θ_4$ και το χαρτί από την $Θ_5$ θα πάει στην $Θ_3$.

Αν είναι στην θέση $Θ_4$, το χαρτί στην $Θ_1$ έχει πιθανότητα $2/3$ να επιλεχθεί, οπότε σε αυτή την περίπτωση ο ρήγας θα μείνει στην θέση του. Αν επιλεχθεί το χαρτί στην $Θ_5$, ο ρήγας μετακινήται στην $Θ_5$.

Αν είναι στην θέση $Θ_5$, έχει πιθανότητα να βρεθέι στην μέση $1/3$. Ακόμα,με πιθανοτήτα $2/3$ μπορεί να επιλεγεί το χαρτί στην $Θ_1$, οπότε σε αυτή την περίπτωση ο ρήγας μένει στην θέση του.

Τελικά ο πίνακας πιθανοτήτων μετάβασης θα είναι:

\begin{equation*}
    P = 
    \begin{pmatrix}
        1/3 & 0 & 2/3 & 0 & 0\\
       2/3 & 1/3 & 0 & 0 & 0\\
        0 & 2/3 & 0 & 1/3 & 0\\
        0 & 0 & 0 & 2/3 & 1/3\\
        0 & 0 & 1/3 & 0 & 2/3\\
    \end{pmatrix}
\end{equation*}

\section*{Άσκηση 7}

Ο χώρος καταστάσεων είναι $\mathbb{X} = \{0, 1, 2, 3, 4, 5\}$. Θέλουμε να φέρουμε 5 φορές συνεχόμενα 6, οπότε η πιθανότητα να πάμε στην επόμενη κατάσταση κάθε φορά είναι $1/6$, ενώ η πιθανότητα να πάμε στην $0$, είναι $5/6$. Αν φτάσουμε στην κατάσταση $5$, το παιχνίδι τελειώνει, όποτε παραμένουμε στην κατάσταση αυτή.

Άρα ο πίνακας πιθανοτήτων μετάβασης θα είναι:

\begin{equation*}
    P = 
    \begin{pmatrix}
        5/6 & 1/6 & 0 & 0 & 0 & 0\\
       5/6 & 0 & 1/6 & 0 & 0 & 0\\
       5/6 & 0 & 0 & 1/6 & 0 & 0\\
        5/6 & 0 & 0 & 0 & 1/6 & 0\\
        5/6 & 0 & 0 & 0 & 0 & 1/6\\
        0 & 0 & 0 & 0 & 0 & 1\\
    \end{pmatrix}
\end{equation*}

Στην περίπτωση που θέλουμε να εμφανιστεί η ακολουθία $65656$, θεωρούμε ότι ο χώρος καταστάσεων είναι ο ίδιος και αναπαριστά σε ποιά θέση της ακολουθίας είμαστε.

Τότε κάθε φορά που φέρνουμε 6 ενώ έπρεπε να φέρουμε 5, γυρνάμε στο πρώτο βήμα, ενώ κάθε άλλη φορά στο μηδενικό.

Έτσι, ο πίνακας πιθανοτήτων μετάβασης θα είναι:

\begin{equation*}
    P = 
    \begin{pmatrix}
        5/6 & 1/6 & 0 & 0 & 0 & 0\\
       4/6 & 1/6 & 1/6 & 0 & 0 & 0\\
       5/6 & 0 & 0 & 1/6 & 0 & 0\\
        4/6 & 1/6 & 0 & 0 & 1/6 & 0\\
        5/6 & 0 & 0 & 0 & 0 & 1/6\\
        0 & 0 & 0 & 0 & 0 & 1\\
    \end{pmatrix}
\end{equation*}

\section*{Άσκηση 8}

Για να είναι η ${X_n}_{n \in \mathbb{N}}$ αλυσίδα \english{Markov}, θα πρέπει να ισχύει ότι για κάθε $n \in \mathbb{N}$ και κάθε $υ_0, \dots, υ_{n-1}, x, y \in \mathbb{X}$ Έχουμε
\begin{equation*}
    \Pr{[X_{n+1} = y | X_0 = υ_0, \dots X_{n-1} = υ_{n-1}, X_n = x]} = \Pr{[X_{n+1} = y | X_n = x]}
\end{equation*}

Πράγματι, εύκολα μπορούμε να δούμε ότι η τιμή του $X_{n+1}$ εξαρτάται μόνο από την τιμή του $X_n$. Ο χώρος καταστάσεων του $X$ είναι$\{0, 1, 2, 3, 4\}$ και κάθε φορά κάνουμε την πράξη: 
\begin{otherlanguage}{english}
    \begin{align*}
        & X_{n+1}  = (S_{n} + \text{dice})\pmod 5 =\\
        & (S_{n}\pmod 5 + \text{dice} \pmod 5) \pmod 5 = \\
        & (X_n + (\text{dice}) \pmod 5) \pmod 5
    \end{align*}
\end{otherlanguage}


όπου \english{dice} είναι το ισοπίθανο γεγονός της ρίψης του ζαριού με χώρο καταστάσεων το $\{1, 2, 3, 4, 5, 6\}$.

Αρα αν $X_n = 0$, τότε ανάλογα με την τιμή της ζαρίας $Z$ έχουμε:
\begin{itemize}
    \item $Z=1$, $X_{N+1} = 1$
    \item $Z=2$, $X_{N+1} = 2$
    \item $Z=3$, $X_{N+1} = 3$
    \item $Z=4$, $X_{N+1} = 4$
    \item $Z=5$, $X_{N+1} = 0$
    \item $Z=6$, $X_{N+1} = 1$
\end{itemize}

Ομοίως για $X_n = 1$,
\begin{itemize}
    \item $Z=1$, $X_{N+1} = 2$
    \item $Z=2$, $X_{N+1} = 3$
    \item $Z=3$, $X_{N+1} = 4$
    \item $Z=4$, $X_{N+1} = 0$
    \item $Z=5$, $X_{N+1} = 1$
    \item $Z=6$, $X_{N+1} = 2$
\end{itemize}

Παρόμοια γίνεται και για τα υπόλοιπα $X_n$.

Έτσι έχουμε τον πίνακα μεταβάσεων:


\begin{equation*}
    P = 
    \begin{pmatrix}
        1/6 & 2/6 & 1/6 & 1/6 & 1/6\\
        1/6 & 1/6 & 2/6 & 1/6 & 1/6 \\
        1/6 & 1/6 & 1/6 & 2/6 & 1/6 \\
        1/6 & 1/6 & 1/6 & 1/6 & 2/6\\
        2/6 & 1/6 & 1/6 & 1/6 & 1/6
    \end{pmatrix}
\end{equation*}

\section*{Άσκηση 11}

Ο χώρος καταστάσεων αποτελείται από τις μεταθέσεις των συμβόλων $\{A, B, C\}$, άρα είναι ο:

$$
\{ABC, ACB, BCA, BAC, CAB, CBA\}
$$

Για τον πίνακα μεταβάσεων έχουμε:
\begin{itemize}
    \item Αν επιλέξουμε το βιβλίο Α, έχουμε με πιθανότητα $p$, ότι θα έχουμε τις ακόλουθες μεταβάσεις:
    \begin{itemize}
        \item Αν $ABC$, τότε $ABC$.
        \item Αν $ACB$, τότε $ACB$.
        \item Αν $BCA$, τότε $ABC$.
        \item Αν $BAC$, τότε $ABC$.
        \item Αν $CAB$, τότε $ACB$.
        \item Αν $CBA$, τότε $ABC$.
    \end{itemize}
    \item Αν επιλέξουμε το βιβλίο B, έχουμε με πιθανότητα $q$, ότι θα έχουμε τις ακόλουθες μεταβάσεις:
    \begin{itemize}
        \item Αν $ABC$, τότε $BAC$.
        \item Αν $ACB$, τότε $BAC$.
        \item Αν $BCA$, τότε $BCA$.
        \item Αν $BAC$, τότε $BAC$.
        \item Αν $CAB$, τότε $BCA$.
        \item Αν $CBA$, τότε $BCA$.
    \end{itemize}
    \item Αν επιλέξουμε το βιβλίο \english{C}, έχουμε με πιθανότητα $r$, ότι θα έχουμε τις ακόλουθες μεταβάσεις:
    \begin{itemize}
        \item Αν $ABC$, τότε $CAB$.
        \item Αν $ACB$, τότε $CAB$.
        \item Αν $BCA$, τότε $CBA$.
        \item Αν $BAC$, τότε $CBA$.
        \item Αν $CAB$, τότε $CAB$.
        \item Αν $CBA$, τότε $CBA$.
    \end{itemize}
\end{itemize}

Τελικά έχουμε τον πίνακα:

\begin{equation*}
    P = 
    \begin{pmatrix}
        p & 0 & 0 & q & r & 0\\
        0 & p & 0 & q & r & 0\\
        p & 0 & q & 0 & 0 & r\\
        p & 0 & 0 & q & 0 & r\\
        0 & p & q & 0 & r & 0\\
        p & 0 & q & 0 & 0 & r
    \end{pmatrix}
\end{equation*}

\section*{Άσκηση 12}
Είναι εύκολα αντιληπτό ότι $\Pr(A) + \Pr(B) = 1$, όπου $\Pr(A), \Pr(B)$ η πιθανότητα μετάβασης από το διαμέρισμα Α στο Β και ανάποδα.

Αρα αν $n$ τα σωματίδια στο διαμέρισμα Α, τότε έχουμε πιθανότητα $\frac{n}{N}$ να επιλεγεί σωματίδιο από το διαμέρισμα Α και να μεταφερθεί στο Β. 

Έτσι μπορούμε να δούμε την διαδικασία αυτή σαν ένα συμμετρικό περίπατο, όπου στο βήμα $n$, μπορεί να μετακινηθεί στο βήμα $n-1$ με πιθανότητα $\frac{n}{N}$ και στο βήμα $n+1$ με πιθανότητα $\frac{N-n}{N}$. 

Στην κατάσταση 0, έχουμε πιθανότητα 1 να μεταφερθούμε στην κατάσταση 1 και στην κατάσταση Ν, έχουμε πιθανότητα 1 να μεταφερθούμε στην $N-1$.

\section*{Άσκηση 13 (Προαιρετική)}


\end{document}